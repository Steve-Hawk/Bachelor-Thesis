We have studied how the gravitational waves propagate in vacuum. 
We now want to understand the relation between the gravitational waves and their source. In most of the physical scenarios the system that produces the gravitational waves is small compared to the distances with the detector. Therefore,
In order to study the solutions of the linearized Einstein's field equation(\ref{linear_einstein_eq})
we make a crucial approximation

\subsection{Solution of the linearized Einstein's field equation}
The generation of gravitational radiation depends on the movements of objects in spacetime. 
So far, we have neglected the presence of matter and we solved the linearized the Einstein's field equation in vaccum. However, if we want to analyze the relation between sources and gravitational waves we need to consider $T_{\mu \nu} \neq 0$ and solve equation(\ref{linear_einstein_eq}):
\[
\square \bar{h}_{\mu \nu} = - 16 \pi \, T_{\mu \nu}
\]
It is possible to solve this equation using a Green function $G(x^{\sigma} -y^{\sigma})$, such that
\beq
\label{green_equation}
\square_{x} G \qty(x^\sigma - y^{\sigma}) =
\delta^{(4)}  \qty(x^\sigma - y^{\sigma})
\eeq
And the general solution is, then, given by
\begin{eqnarray}
\label{h_green_function}
\bar{h}_{\mu \nu} (x^{\sigma}) =
-16 \pi \int
 G \qty(x^\sigma - y^{\sigma})
 T_{\mu \nu} (y^{\sigma}) 
 \, \dd[4] y 
 \\
 \square_{x} \bar{h}_{\mu \nu} (x^{\sigma}) 
 = 
-16 \pi \int   \square_{x}  G \qty(x^\sigma - y^{\sigma})
 T_{\mu \nu} (y^{\sigma}) 
 \, \dd[4] y 
 =
 - 16 \pi \, T_{\mu \nu} (x^{\sigma})
 \notag
\end{eqnarray}
There are two solutions of equation(\ref{green_equation}): one solution represents a wave travelling forward in time and, the other represents a wave travelling backward in time. The two solutions are called, respectively, retarded and advanced. We are interested in the retarded Green function, which represents the accumulated effect of signals received at $(x^{0},x^{1},x^{2},x^{3})$ from a source at $(y^{0},y^{1},y^{2},y^{3})$: %mathematical methods in physics
\[
G \qty(x^\sigma - y^{\sigma}) =
-\dfrac{1}{4 \pi\abs{\vb{x} - \vb{y}}} \, \mathlarger{\delta} \qty[\abs{\vb{x} - \vb{y}} - (x^{0}- y^{0})]
\, \theta(x^{0} -y^{0})
\]
where have used boldface to denote the patial vectors $\vb{x} = (x^{1}, x^{2}, x^{3})$ and $\vb{y} =(y^{1}, y^{2}, y^{3})$, with norm $\abs{\vb{x}-\vb{y}} = \qty[\delta_{ij}(x^{i}-y^{i})(x^{j}-y^{j})]^{1/2}$. The Heaviside step function $\theta(x^{0} - y^{0})$ is 1 when $x^{0}>y^{0}$, and zero otherwise. \\
Plugging the retarded Green function into equation(\ref{h_green_function}) 
\beq
\label{h_retarded_solution}
\bar{h}_{\mu \nu} (t,\vb{x}) = 4 \int
\dfrac{1}{\abs{\vb{x}- \vb{y}}} T_{\mu \nu} \qty(t-\abs{\vb{x}-\vb{y}},\vb{y}) \, \dd[3]y 
\eeq
where $t=x^{0}$ and the integration is made over the spatial coordinates. 
From equation(\ref{h_retarded_solution}) we notice that the metric perturbation is influenced by the matter and energy distribution, $T_{\mu \nu}$, at time $t-\abs{\vb{x}-\vb{y}}$. 
Since the gravitational radiation travels at the speed of light $c=1$, the metric perturbation at $(t,\vb{x})$ is influenced by the radiation that was produced by the source at the retarded time $t_r = t-\abs{\vb{x}-\vb{y}}$.\\
We have obtained a general solution, however it possible to derive a formula that reaveals the quadrupole nature of the gravitational radiation if we assume:
\begin{itemize}
\item \textbf{far field approximation}: the metric perturbation (\ref{h_retarded_solution}) is evaluated at large distances from the source
\begin{equation}
\label{far_field_approx}
\abs{\vb{x}-\vb{y}} \approx \abs{x} \equiv r
\end{equation}
The fractional error of this approximation scales as $\sim L/r$, where $L$ is the size of the source.

\item \textbf{slowly moving source}: the light traverses the source much faster than the components of the cource itself do. Therefore, the source moves at non relativistic speeds.

\item \textbf{isolated system}: the source of the gravitational radiation is an isolated and compact. We assume that our system and the radiation are not gravitationally influenced by other bodies.

\end{itemize}
So we rewrite equation(\ref{h_retarded_solution}) using the far field approximation:
\[
\bar{h}_{\mu \nu} (t,\vb{x}) = \dfrac{4}{r} \int
 T_{\mu \nu} \qty(t-r,\vb{y}) \, \dd[3]y
\]
Using the Fourier trassform and inverse with respect to time
\bea
\phi (t,\vb{x}) &=&  \mathcal{F}^{-1}[\tilde{\phi}(\omega , \vb{x})] \equiv 
\dfrac{1}{\sqrt{2 \pi}} \, \int \dd \omega e^{-i\omega t} \tilde{\phi}(\omega , \vb{x}) \\
\tilde{\phi} (\omega,\vb{x}) &=&  \mathcal{F}[\phi(t , \vb{x})] \equiv 
\dfrac{1}{\sqrt{2 \pi}} \, \int \dd t e^{i\omega t} \phi (\omega , \vb{x})
\eea
applied to the metric perturbation 
\begin{eqnarray}
\h _{\mu \nu}(\omega , t) 
&\equiv &
\mathcal{F}[\bar{h} _{\mu \nu}(t,\vb{x})]
=
\dfrac{1}{\sqrt{2 \pi}} \, \int  e^{i\omega t} \bar{h}_{\mu \nu} (t , \vb{x}) \; \dd t
\notag
\\
&=&
\dfrac{4}{\sqrt{2 \pi}} \, \int  e^{i\omega t}  \dfrac{T_{\mu \nu}(t_r,\vb{y})}{r} \; \dd t \, \dd ^{3} y
\notag
\\
&=&
\dfrac{4}{\sqrt{2 \pi} \, r} 
\, \int  e^{i\omega (t_r + r)} \, T_{\mu \nu}(t_r,\vb{y}) \; \dd t_r \, \dd ^{3} y
\notag
\\
&=&
\label{F(h)}
\dfrac{4\,  e^{i\omega r}}{r} \int  \, \ten _{\mu \nu}(\omega,\vb{y}) \;  \, \dd ^{3} y
\end{eqnarray}
where we used a change of variable and we defined the Fourier transform of the energy momentum tensor as $\ten _{\mu \nu} \equiv \mathcal{F}[T_{\mu \nu}]$. \\
The Lorenz gauge condition $\partial_{\mu } \bar{h} ^{\mu \nu} =0$ in the Fourier space becomes
\bea
\mathcal{F}\qty[\partial_{0} \bar{h} ^{0 \nu} + \partial_{j} \bar{h} ^{j \nu}] =0
\\
\h ^{0 \nu} = \dfrac{i}{\omega} \partial_{j} \h ^{j \nu}
\eea
As a consequence, we only need to calculate the spacelike components $\h^{j \nu}$.
We set $\nu=k$ in order to find $\h ^{0 k}$ from $\h ^{j k}$, afterwards we use $\h ^{k 0}$ to get $h ^{0 0}$.
The integration by parts of the spacelike components of equation(\ref{F(h)}) is 
\[
\int \ten ^{j k} \dd ^{3} y =
\int \partial_m \qty( \ten ^{m k} y^{j})
\dd ^{3} y
- 
\int \partial_m \qty( \ten ^{m k}) y^{j}
\dd ^{3} y
\]
Since we assumed that the source is isolated, the first term, which is a surface integral, vanishes.
Whereas, the conservation of the energy-momentum tensor $\partial_\mu T^{\mu \nu}=0$ yields in the Fourier space
\[
- \partial_m \qty( \ten ^{m k})  = i \omega \ten ^{0 k}
\] 
Thus,
\bea
\int \ten ^{j k} (\omega , \vb{y}) \dd ^{3} y 
&=&
i \omega \int y^{j} \ten ^{0 k} \; \dd ^{3} y \\
\text{symmetry of } \ten_{k \nu} \rightarrow &=& 
\dfrac{i \omega}{2} \int \qty(
y^{j} \ten ^{0 k} + y^{k} \ten ^{0 j}
) \dd ^{3} y
\\
\partial _l \qty(y^{k} y^{j} \ten ^{0 l}) = \delta ^{k} _{l} y^{j} \ten ^{0l}+
\delta ^{j} _{l} y^{k} \ten ^{0l}+ 
y^{k} y^{j}  \partial _l \ten ^{0 l}
\rightarrow
&=&
\dfrac{i \omega}{2} \int 
\qty[
\partial _l \qty(y^{k} y^{j} \ten ^{0 l})
- 
y^{k} y^{j}  \partial _l \ten ^{0 l}
]
\, \dd ^{3} y
\\
 \partial _l \ten ^{0 l} = \partial _l \ten ^{ l 0} = -i\omega \ten ^{00} \rightarrow &=&
 -
\dfrac{\omega^{2}}{2}
\int
y^{k} y^{j} 
\ten ^{00} (\omega , \vb{y}) \;
\dd ^{3} y
\\
\eea
Then, equation(\ref{F(h)}) becomes
\bea
\h _{k j} &=& 
-\dfrac{4 \, e^{i \omega r}}{r}  
\dfrac{\omega^{2}}{2}
\int
y^{k} y^{j} 
\ten ^{00} (\omega , \vb{y}) \;
\dd ^{3} y
\\
\mathcal{F}\qty[\pdv[2]{T^{00}}{t}] = - \omega ^2 \mathcal{F}[T^{00}]
\rightarrow
&=&
\dfrac{2}{r} \int y^{k} y^{j}
\mathcal{F}\qty[\pdv[2]{}{t}T^{00}(t_r,\vb{y})]
\; 	\dd[3] y 
\\
&=&
\mathcal{F} \qty[\dfrac{2}{r}\, \pdv[2]{}{t} \qty(
 \int
y^{k} y^{j} 
T ^{00} (t_r , \vb{y}) \;
\dd ^{3} y
)
]
\eea
We can transform back the above result to obtain the original metric perturbation
\beq
\label{h_bar_quadrupole_formula}
\bar{h}_{kj} = \dfrac{2}{r} \dv[2]{}{t} I_{kj} (t_r)
\eeq
where we defined the \textbf{quadrupole moment tensor} 
\begin{equation}
\label{quadrupole_moment_tensor}
I_{k j} (t) =
 \int
y_{k} y_{j} 
T ^{00} (t , \vb{y}) \;
\dd ^{3} y
\end{equation}
To complete the derivation we need to express the metric perturbation in the TT gauge, so we must make the right hand side of equation(\ref{h_bar_quadrupole_formula}) traceless and transverse.\\
We begin by introducing the spatial projection tensor
\begin{equation}
\label{projection_tensor}
P_{i j} = \delta_{i j} - n_{i} n_{j}
\end{equation}
which projects the components of a tensor (with rank 2) into a surface orthogonal to the unit vector $n^{i}$
\[
(P_{i j} X^{i l}) n^{j} = X^{ j l} n_{j}-  n_{i} n_{j} X^{i l} n^{j} = 0
\]
We can us the \textbf{projection tensor} to construct the transverse-traceless version of a symmetric spatial tensor $X_{ij}$ via
\begin{equation}
\label{TT_projection}
X^{TT} _{ij} = \qty(:P_{i}^{k}::P_{j}^{l}: - \dfrac{1}{2} P_{ij} P^{kl}) X_{kl}
\end{equation}
where the first and second terms make the tensor, respectively, transverse and tracless.\\
In addition, we define the \textbf{reduced quadrupole moment tensor} as
\begin{equation}
\label{quadrupole_moment_tensor}
\mathcal{I}_{kj} = I_{kj} - \dfrac{1}{3} \delta_{kj} I \hspace{1.5cm} \text{where } I=\eta^{ l m} I_{l m} = I^{m} _{m}
\end{equation}
which is traceless, and, for $T^{00} = \rho$, it assume the expression
\[
\mathcal{I} _{kj} = \int \rho(\vb{y}) \qty(y_{k} y_{j} - \dfrac{1}{3}\delta_{kj} y^{l} y_{l} ) \; \dd[3] y
\]
We now have all the concepts to write down 
the \textbf{quadrupole formula}
\begin{equation}
\label{h_TT_quadrupole_formula}
h^{\T}_{ij} = \dfrac{2}{r} \dv[2]{\mathcal{I}_{kl}(t_r)}{t} \qty(:P_{i}^{k}::P_{j}^{l}: - \dfrac{1}{2} P_{ij} P^{kl})
\end{equation}
which represents the metric perturbation of equation(\ref{h_bar_quadrupole_formula}) in the TT gauge.