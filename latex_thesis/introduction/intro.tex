The constancy of the speed of light as measured by observers in different reference frames forces space and time to be mixed into spacetime.
When Albert Einstein developed the General Theory of Relativity, he endowed spacetime with curvature and made it dynamical, creating a theory in which all predictions for physical measurements are invariant under changes in coordinates.
The speed of light is independent of the observer and it is the speed of causality.
Therefore, gravity must be casual as well, and
any change to a gravitating source is communicated to a distant observer no faster than the speed of light $c$.
Such communication of a change in spacetime from a source to an observer leads to the idea that there must exists some notion of gravitational radiation.\\
From the General Theory of Relativity, it is possible to obtain a tensorial wave equation, which governs the behavior of such gravitational radiation.
Therefore, we propose in  section(\ref{from_equation_to_solution}) the derivation of the gravitational wave solution from the Einstein's field equation using the linearized approximation.
Applying the transverse traceless gauge, we find out that the number of raditative degrees of freedom of the general theory of relativity is two.\\
The triumph of theoretical prediction of the gravitational waves (GW) is confirmed by the several experiments.\\
The first indirect evidence of the existence of gravitational waves was given by Hulse and Taylor in 1993 \cite{weisberg_relativistic_2004,weisberg_timing_2010}. 
From the study of the orbital decay of a binary pulsar, Hulse and Taylor inferred the emission of gravitational waves, since the two neutron stars were spiraling together at just the rate predicted by general relativity’s theory of gravitational radiation reaction.\\
Subsequently, in 2016, the laser interferometer gravitational wave detector LIGO measured directly the signal of a gravitational wave produced by a binary black hole \cite{abbott_observation_2016}.
And, one year later LIGO-Virgo detector network observed a gravitational-wave signal from the inspiral of two low-mass compact objects consistent with a binary neutron star (BNS) merger \cite{abbott_gw170817:_2017}.\\
So, in order to understand the basic principles of gravitational wave detectors, in section(\ref{effects_gw}) we study how the gravitational waves perturb free falling particles.\\
Then, by studying a ring of free falling test particles, we explain the physical interpretation of the two radiative degrees of freedom two polarization states of the GWs.\\
The gravitational waves are not only the triumph of the General Relativity, but they also give us a completely new method to investigate the nature of binary systems. 
Binary neutron stars and black holes can imprint a large amount of information in the produced gravitational radiations.
However, a thorough study of a gravitational signals needs to be combined with accurate numerical simulations of binary mergers in order to fully understand the link between the  gravitational wave production and their sources.\\
So, firstly, we find a simple theoretical model by solving the linearized Einstein's field equation with the presence of matter in section(\ref{production_gw}).
Then, we apply the obtained solution, the so-called quadrupole formula, for a slowly-moving binary source.
And, finally, in the last section(\ref{numerical_evolution}), by using the open-source computational infrastructure Einstein Tookit\cite{loffler_einstein_2012}, we study the relativistic numerical simulations of compact binaries: binary black holes (BBH) with six different quasi-equilibrium initial configurations and a binary neutron star (BNS) is accomplished.
The theoretical prediction from the quadrupole formula, that the frequency of the gravitational wave is twice the orbital angular frequncy of the binary, is confirmed in the simulations through a Fourier analysis of the gravitational signal.
In addition, we propose a simple gravitational wave extraction method and we also analyze the rest mass density of the binary neutron stars in order to find the relation with the gravitational signal behavior.\\

\textbf{NOTATION}\\
Throughout, we will use a spacelike signature $(-,+,+,+)$ and a system of geometrised units in which $G=c=1$, however depending on the need we will also indicate explicitely the speed of light $c=299\,792\,458 \,\metre \per \second$, and the Newton constant of gravitation $G=6.67408 \times 10^{-11} \, \newton \usk \metre^2 \usk \kilogram^{-2}$ \cite{codata_blog_codata_nodate}.\\
We use the Einstein summation convention for  reapeted indeces.
Greek letters are used for summing over all the indeces from 0 up to 3,
whereas latin letters sum only on the spatial indeces 1 2 and 3.\\
We use the controviariant notation of a four-vector
\[
x^{\sigma} = (x^{0},x^{1}, x^{2},x^{3})
\]
or, when we want to aid the comparison with classical Newtonian expression, we use boldface to denote the spatial vectors $\vb{x} =(x,y,z)$ and we rewrtie the four-vectors as 
\[
x^{\sigma} = (x^{0},x^{1}, x^{2},x^{3}) = (t,\vb{x}) = (t,x,y,z)
\]
The four-dimensional covariant and partial derivatives will be indicated respectively with $\nabla _{\mu}$ and $\partial _\mu$.\\

\textbf{Codes, Data and Reproducibility of the simulations}\\
The codes used for the numerical simulations, the scripts for the data analysis and the scripts used to generate the figures are available at the website
\\
\url{https://github.com/lorenzsp/thesis}
\\
In addition, we have also included the data produced by the simulations in order to reproduce all the results that we have reported in this paper. \\
In order to reproduce the simulations of the binary black holes and of the binary neutron star, it is possible to download the Einstein Tookit from the website \cite{EinsteinToolkit:web}, and use the parameter files at the website \url{https://github.com/lorenzsp/thesis}.