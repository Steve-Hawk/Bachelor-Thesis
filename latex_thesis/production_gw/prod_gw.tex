We have studied how the gravitational waves propagate in vacuum. 
We now want to understand the relation between the gravitational waves and their source. In most of the physical scenarios the system that produces the gravitational waves is small compared to the distances with the detector. Therefore,
in order to study the solutions of the linearized Einstein's field equation(\ref{linear_einstein_eq})
we make a crucial approximation

\subsection{Solution of the linearized Einstein's field equation}
\label{solution_linearized_einstein_eq}
The generation of gravitational radiation depends on the movements of objects in spacetime. 
So far, we have neglected the presence of matter and we solved the linearized Einstein's field equation in vacuum. However, if we want to analyze the relation between sources and gravitational waves we need to consider $T_{\mu \nu} \neq 0$ and solve equation(\ref{linear_einstein_eq}):
\[
\square \bar{h}_{\mu \nu} = - 16 \pi \, T_{\mu \nu}
\]
It is possible to solve this equation using a Green function $G(x^{\sigma} -y^{\sigma})$, such that
\beq
\label{green_equation}
\square_{x} G \qty(x^\sigma - y^{\sigma}) =
\delta^{(4)}  \qty(x^\sigma - y^{\sigma})
\eeq
And the general solution is, then, given by
\begin{eqnarray}
\label{h_green_function}
\bar{h}_{\mu \nu} (x^{\sigma}) =
-16 \pi \int
 G \qty(x^\sigma - y^{\sigma})
 T_{\mu \nu} (y^{\sigma}) 
 \, \dd[4] y 
 \\
 \square_{x} \bar{h}_{\mu \nu} (x^{\sigma}) 
 = 
-16 \pi \int   \square_{x}  G \qty(x^\sigma - y^{\sigma})
 T_{\mu \nu} (y^{\sigma}) 
 \, \dd[4] y 
 =
 - 16 \pi \, T_{\mu \nu} (x^{\sigma})
 \notag
\end{eqnarray}
There are two solutions of equation(\ref{green_equation}): one solution represents a wave travelling forward in time and, the other represents a wave travelling backward in time. 
The two solutions are called, respectively, retarded and advanced. 
We are interested in the \textbf{retarded Green function}, which represents the accumulated effect of signals received at $(x^{0},x^{1},x^{2},x^{3})$ from a source at $(y^{0},y^{1},y^{2},y^{3})$: %mathematical methods in physics
\[
G \qty(x^\sigma - y^{\sigma}) =
-\dfrac{1}{4 \pi\abs{\vb{x} - \vb{y}}} \, \mathlarger{\delta} \qty[\abs{\vb{x} - \vb{y}} - (x^{0}- y^{0})]
\, \theta(x^{0} -y^{0})
\]
where have used boldface to denote the patial vectors $\vb{x} = (x^{1}, x^{2}, x^{3})$ and $\vb{y} =(y^{1}, y^{2}, y^{3})$, with norm $\abs{\vb{x}-\vb{y}} = \qty[\delta_{ij}(x^{i}-y^{i})(x^{j}-y^{j})]^{1/2}$. The Heaviside step function $\theta(x^{0} - y^{0})$ is 1 when $x^{0}>y^{0}$, and zero otherwise. \\
Plugging the retarded Green function into equation(\ref{h_green_function}) and integrating on the $y^{0}$ coordinate we obtain
\beq
\label{h_retarded_solution}
\bar{h}_{\mu \nu} (t,\vb{x}) = 4 \int
\dfrac{1}{\abs{\vb{x}- \vb{y}}} T_{\mu \nu} \qty(t-\abs{\vb{x}-\vb{y}},\vb{y}) \, \dd[3]y 
\eeq
where $t=x^{0}$ and the integration is made over the spatial coordinates. 
From equation(\ref{h_retarded_solution}) we notice that the metric perturbation is influenced by the matter and energy distribution, $T_{\mu \nu}$, at time $t-\abs{\vb{x}-\vb{y}}$. 
Since the gravitational radiation travels at the speed of light $c=1$, the metric perturbation at $(t,\vb{x})$ is influenced by the radiation that was produced by the source at the retarded time $t_r = t-\abs{\vb{x}-\vb{y}}$.\\
We have obtained a general solution, however it possible to derive a formula that reaveals the quadrupole nature of the gravitational radiation if we assume:
\begin{itemize}
\item \textbf{far field approximation}: the metric perturbation (\ref{h_retarded_solution}) is evaluated at large distances from the source
\begin{equation}
\label{far_field_approx}
\abs{\vb{x}-\vb{y}} \approx \abs{x} \equiv r
\end{equation}
The fractional error of this approximation scales as $\sim L/r$, where $L$ is the size of the source.

\item \textbf{slowly moving source}: the light traverses the source much faster than the components of the cource itself do. Therefore, the source moves at non relativistic speeds.

\item \textbf{isolated system}: the source of the gravitational radiation is an isolated and compact. We assume that our system and the radiation are not gravitationally influenced by other bodies.

\end{itemize}
So we rewrite equation(\ref{h_retarded_solution}) using the far field approximation:
\[
\bar{h}_{\mu \nu} (t,\vb{x}) = \dfrac{4}{r} \int
 T_{\mu \nu} \qty(t-r,\vb{y}) \, \dd[3]y
\]
Since most of the sources are very far from the detection point, the above result is a very good approximation in most of the cases, and it shows the $1/r$ dependency of the gravitational wave.\\
Using the Fourier transform and inverse with respect to time
\bea
\phi (t,\vb{x}) &=&  \mathcal{F}^{-1}[\tilde{\phi}(\omega , \vb{x})] \equiv 
\dfrac{1}{\sqrt{2 \pi}} \, \int \dd \omega e^{-i\omega t} \tilde{\phi}(\omega , \vb{x}) \\
\tilde{\phi} (\omega,\vb{x}) &=&  \mathcal{F}[\phi(t , \vb{x})] \equiv 
\dfrac{1}{\sqrt{2 \pi}} \, \int \dd t e^{i\omega t} \phi (\omega , \vb{x})
\eea
applied to the metric perturbation 
\begin{eqnarray}
\h _{\mu \nu}(\omega , t) 
&\equiv &
\mathcal{F}[\bar{h} _{\mu \nu}(t,\vb{x})]
=
\dfrac{1}{\sqrt{2 \pi}} \, \int  e^{i\omega t} \bar{h}_{\mu \nu} (t , \vb{x}) \; \dd t
\notag
\\
&=&
\dfrac{4}{\sqrt{2 \pi}} \, \int  e^{i\omega t}  \dfrac{T_{\mu \nu}(t_r,\vb{y})}{r} \; \dd t \, \dd ^{3} y
\notag
\\
&=&
\dfrac{4}{\sqrt{2 \pi} \, r} 
\, \int  e^{i\omega (t_r + r)} \, T_{\mu \nu}(t_r,\vb{y}) \; \dd t_r \, \dd ^{3} y
\notag
\\
&=&
\label{F(h)}
\dfrac{4\,  e^{i\omega r}}{r} \int  \, \ten _{\mu \nu}(\omega,\vb{y}) \;  \, \dd ^{3} y
\end{eqnarray}
where we used a change of variable and we defined the Fourier transform of the energy momentum tensor as $\ten _{\mu \nu} \equiv \mathcal{F}[T_{\mu \nu}]$. \\
The Lorenz gauge condition $\partial_{\mu } \bar{h} ^{\mu \nu} =0$ in the Fourier space becomes
\bea
\mathcal{F}\qty[\partial_{0} \bar{h} ^{0 \nu} + \partial_{j} \bar{h} ^{j \nu}] =0
\\
\h ^{0 \nu} = \dfrac{i}{\omega} \partial_{j} \h ^{j \nu}
\eea
As a consequence, we only need to calculate the spacelike components $\h^{j \nu}$.
We set $\nu=k$ in order to find $\h ^{0 k}$ from $\h ^{j k}$, afterwards we use $\h ^{k 0}$ to get $h ^{0 0}$.
The integration by parts of the spacelike components of equation(\ref{F(h)}) is 
\[
\int \ten ^{j k} \dd ^{3} y =
\int \partial_m \qty( \ten ^{m k} y^{j})
\dd ^{3} y
- 
\int \partial_m \qty( \ten ^{m k}) y^{j}
\dd ^{3} y
\]
Since we assumed that the source is isolated, the first term, which is a surface integral, vanishes.
Whereas, the conservation of the energy-momentum tensor $\partial_\mu T^{\mu \nu}=0$ yields in the Fourier space
\[
- \partial_m \qty( \ten ^{m k})  = i \omega \ten ^{0 k}
\] 
Notice that the conservation of the energy-momentum tensor is a very strong assumption, because the motion of bodies is governed by non-gravitational interactions. 
However, and remarkably, the result depends only on the sources motion and not on the forces acting on them.
Thus,
\bea
\int \ten ^{j k} (\omega , \vb{y}) \dd ^{3} y 
&=&
i \omega \int y^{j} \ten ^{0 k} \; \dd ^{3} y \\
\text{symmetry of } \ten_{k \nu} \rightarrow &=& 
\dfrac{i \omega}{2} \int \qty(
y^{j} \ten ^{0 k} + y^{k} \ten ^{0 j}
) \dd ^{3} y
\\
\partial _l \qty(y^{k} y^{j} \ten ^{0 l}) = \delta ^{k} _{l} y^{j} \ten ^{0l}+
\delta ^{j} _{l} y^{k} \ten ^{0l}+ 
y^{k} y^{j}  \partial _l \ten ^{0 l}
\rightarrow
&=&
\dfrac{i \omega}{2} \int 
\qty[
\partial _l \qty(y^{k} y^{j} \ten ^{0 l})
- 
y^{k} y^{j}  \partial _l \ten ^{0 l}
]
\, \dd ^{3} y
\\
 \partial _l \ten ^{0 l} = \partial _l \ten ^{ l 0} = -i\omega \ten ^{00} \rightarrow &=&
 -
\dfrac{\omega^{2}}{2}
\int
y^{k} y^{j} 
\ten ^{00} (\omega , \vb{y}) \;
\dd ^{3} y
\\
\eea
Then, equation(\ref{F(h)}) becomes
\bea
\h _{k j} &=& 
-\dfrac{4 \, e^{i \omega r}}{r}  
\dfrac{\omega^{2}}{2}
\int
y^{k} y^{j} 
\ten ^{00} (\omega , \vb{y}) \;
\dd ^{3} y
\\
\mathcal{F}\qty[\pdv[2]{T^{00}}{t}] = - \omega ^2 \mathcal{F}[T^{00}]
\rightarrow
&=&
\dfrac{2}{r} \int y^{k} y^{j}
\mathcal{F}\qty[\pdv[2]{}{t}T^{00}(t_r,\vb{y})]
\; 	\dd[3] y 
\\
&=&
\mathcal{F} \qty[\dfrac{2}{r}\, \pdv[2]{}{t} \qty(
 \int
y^{k} y^{j} 
T ^{00} (t_r , \vb{y}) \;
\dd ^{3} y
)
]
\eea
We can transform back the above result to obtain the original metric perturbation
\beq
\label{h_bar_quadrupole_formula}
\bar{h}_{kj} = \dfrac{2}{r} \dv[2]{}{t} I_{kj} (t_r)
\eeq
where we defined the \textbf{quadrupole moment tensor} 
\begin{equation}
\label{quadrupole_moment_tensor}
I_{k j} (t) =
 \int
y_{k} y_{j} 
T ^{00} (t , \vb{y}) \;
\dd ^{3} y
\end{equation}
To complete the derivation we need to express the metric perturbation in the TT gauge, so we must make the right hand side of equation(\ref{h_bar_quadrupole_formula}) traceless and transverse.\\
We begin by introducing the spatial projection tensor
\begin{equation}
\label{projection_tensor}
P_{i j} = \delta_{i j} - n_{i} n_{j}
\end{equation}
which projects the components of a tensor (with rank 2) into a surface orthogonal to the unit vector $n^{i}$
\[
(P_{i j} X^{i l}) n^{j} = X^{ j l} n_{j}-  n_{i} n_{j} X^{i l} n^{j} = 0
\]
We can us the \textbf{projection tensor} to construct the transverse-traceless version of a symmetric spatial tensor $X_{ij}$ via
\begin{equation}
\label{TT_projection}
X^{TT} _{ij} = \qty(:P_{i}^{k}::P_{j}^{l}: - \dfrac{1}{2} P_{ij} P^{kl}) X_{kl}
\end{equation}
where the first and second terms make the tensor, respectively, transverse and tracless.\\
In addition, we define the \textbf{reduced quadrupole moment tensor} as
\begin{equation}
\label{reduced_quadrupole_moment_tensor}
\mathcal{I}_{kj} = I_{kj} - \dfrac{1}{3} \delta_{kj} I \hspace{1.5cm} \text{where } I=\eta^{ l m} I_{l m} = I^{m} _{m}
\end{equation}
which is traceless, and, for $T^{00} = \rho$, it assume the expression
\[
\mathcal{I} _{kj} = \int \rho(\vb{y}) \qty(y_{k} y_{j} - \dfrac{1}{3}\delta_{kj} y^{l} y_{l} ) \; \dd[3] y
\]
We now have all the concepts to write down 
the \textbf{quadrupole formula}
\begin{equation}
\label{h_TT_quadrupole_formula}
h^{\T}_{ij} = \dfrac{2}{r} \dv[2]{\mathcal{I}_{kl}(t_r)}{t} \qty(:P_{i}^{k}::P_{j}^{l}: - \dfrac{1}{2} P_{ij} P^{kl})
\end{equation}
which represents the metric perturbation of equation(\ref{h_bar_quadrupole_formula}) in the TT gauge, since $h^{\T}_{\mu \nu} = \bar{h} ^{\T}_{\mu \nu}$.
Notice that the gravitational wave scales as $\sim 1/r$.
\\


\subsection{The nature of the gravitational radiation}
The quadrupole formula (\ref{h_TT_quadrupole_formula}) and its derivation gives a  first insight into the properties of the gravitational waves and their sources.\\
Firstly, the gravitational radiation has a \textbf{quadrupolar nature}, because the GW produced by an isolated nonrelativisitc object is proportional to the second derivative of the reduced quadrupole moment of the energy density. 
It is possible to justify qualitatively
the quadrupole nature, defining the gravitational analogue of the dipole moment: \textbf{mass dipole moment}
\begin{equation}
\label{mass_dipole_moment}
\vb{D} = \sum _{i} m_{i} \vb{x}_i
\end{equation}
where the $m_i$ is the rest mass and $\vb{x}_{i}$ is the spatial postion of particle $i$.\\
The leading contribution to the electromagnetic radiation comes from the changing dipole moment. 
However, the first derivative of the mass dipole moment is the total linear momentum
\[
\dv{\vb{D}}{t} = \sum _{i} m_i \dv{\vb{x}_i}{t} = \vb{p}
\]
Since the total linear momentum is conserved, there can be no mass dipole radiation from any source.\\
Similiarly, the gravitational analogue of the magnetic dipole moment is 
\[
\vb{\mu} = \sum_{i} \vb{x}_i \times \qty(m_i \dv{\vb{x}_i}{t}) =\vb{J}
\]
where $\vb{J}$ is the total angular momentum of the system. 
Since the total angular momentum is conserved. Hence, there can be no dipole radiation of any sort from a gravitational source.
It should be stressed that for a spherical  distribution of matter (or energy) the quadrupole moment is a constant, even if the body is rotating: thus, a spherical star does not emit gravitational waves.\\
We now study the GW-emission of a binary system in circular orbit with radius $R$.
We assume that two equal-mass stars are orbiting far from each others with an angular frequency $\omega$ and they can be trated as point particles on the $x^{1}-x^{2}$ plane. Thus,
\[
T^{00} (t,\vb{x}) = M \delta(x^{3}) \qty[
\delta\qty(x^{1}-R\cos \omega t)
\delta\qty(x^{2}-R\sin \omega t)
+
\delta\qty(x^{1}+R\cos \omega t)
\delta\qty(x^{2}+R\sin \omega t)
]
\]
The motion of the system is studied using the Newtonian approximations, so using the  Kepler third law we obtain the link between the angular frequency and the radius of the orbit:
\[
\omega ^{2} 
a^{3} = M_T G 
\qquad
\rightarrow
\qquad 
\omega = \qty(\dfrac{M}{4R^{3}})^{1/2}
\]
where we used as total mass of the system $M_T = 2M$, semi-major axis $a=2R$ and $G=1$.\\
The quadrupole moment tensor (\ref{quadrupole_moment_tensor}) becomes
\[
I_{ij}(t) =
2 M R^{2}
\begin{bmatrix}
 \cos ^{2} \omega t  &
 \cos \omega t \sin \omega t &
0
\\
 \cos \omega t \sin \omega t &
 \sin ^{2} \omega t &
0
\\
0 & 0 & 0\\
\end{bmatrix} 
\]
Then, the reduced quadrupole moment (\ref{reduced_quadrupole_moment_tensor}) is easily found to be
\[
\mathcal{I} _{ij}(t) =
2 M R^{2}
\begin{bmatrix}
 \cos ^{2} \omega t -1/3  &
 \cos \omega t \sin \omega t &
0
\\
 \cos \omega t \sin \omega t &
 \sin ^{2} \omega t -1/3&
0
\\
0 & 0 & -1/3\\
\end{bmatrix} 
\]
Taking the second derivative of the above tensor and using the projection tensor with $n_{j} =\delta_{j3} \rightarrow \vb{n} = (0,0,1)$:
\[
P_{jk} = \delta_{jk}- n_{j} n_{k}
\]
we obtain the gravitational wave through the the quadrupole formula(\ref{h_TT_quadrupole_formula}):
\[
h^{\T} _{ij} = \dfrac{8\,G M \,R^{2} \omega^{2}}{c^{4}\, r} \,
\begin{bmatrix}
- \cos 2 \omega t   &
 - \sin 2 \omega t &
0
\\
  \sin 2 \omega t &
 \cos 2 \omega t &
0
\\
0 & 0 & 0\\
\end{bmatrix}
\]
where we inserted $G$ and $c$ in order to give an estimates of the coefficient.\\
Two remarkable aspects of the above formula are that the gravitational radiation is emitted at  twice the angular frequency at which the system rotates, and $h_{+} ^{\T}= i h_{\times} ^{\T}$ so the wave is circularly polarized.\\
We will see again these aspects when we will treat the numerical simulations, but let us now give an order-of-magnitude estimate of the coefficient
\[
\mathcal{H} = 
 \dfrac{8\,G M \,R^{2} \omega^{2}}{c^{4}\, r} \,
\]
We assume that the two objects to be separated by a distance $R$ equal ten times their Schwarzschild radii $r_s=2 G M/c^{2}$.
In addition we consider that the two objects have approximately the mass of the Sun $M=M_{\odot}=2\times 10^{30} \kilo \gram $ and they rotate with an angular frequency given by the Kepler's third law $\omega = \sqrt{GM/(4R^{3})}$.
\bea
\dfrac{G}{c^{4}} = 8.26 \times 10^{-45} \kilo \gram^{-1} \, \second ^{2} \meter ^{-1} 
\\
r_s=2.95 \times 10^{3} \, \meter
\\
R  = 10 r_s = 2.95 \times 10 ^{4} \, \meter
\\
\omega = 1.13 \time 10 ^{3} \second ^{-1}
\eea
Pluging into $\mathcal{H}$ and considering the source at the a cosmological distance $r= 100 \, \mega pc \approx 3.09 \times 10^{24} \meter$ we have
\[
\mathcal{H} = 4.79 \times 10^{-23}
\]


