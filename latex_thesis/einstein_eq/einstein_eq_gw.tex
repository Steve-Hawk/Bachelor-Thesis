Solving the Einstein's field equation is a mathematically complicated problem, because in general it is difficult to find solutions to the field equations which are non-linear    \cite{weinsteinEinsteinDiscoveryGravitational2016}.
Therefore, we will obtain the gravitational wave solutions using the weak field linearized approximation and we will discuss the  gauge transformations that lead to the two radiative degrees of freedom of the Einsten's field equation.\\

\subsection{Linearized Einstein's field equation}
The Einstein's field equation is a tensor equation that represents how the geometry of spacetime is related to the presence of masses and energy:
\begin{equation}
\label{einstein_eq}
G_{\mu \nu} \equiv R_{\mu \nu} - \dfrac{1}{2} \,  g_{\mu \nu} \, R =8 \, \pi \, T_{\mu \nu}
\end{equation}
On the right hand side we have the energy-momentum tensor $T_{\mu \nu}$ (or stress-energy tesnor), which is interpreted as the flux of four momentum $p^\mu$ accross a surface of constant $x^\nu$. 
On the left hand side the Einstein tensor $G_{\mu \nu}$ includes a measure of the curvature of spacetime through the Ricci tensor $R_{\mu \nu}$, the Ricci scalar $R = R_{\mu \nu} g^{\mu \nu} $ and the metric $g_{\mu \nu}$. \\
In order to solve the Einstein's equation we will make use of the weak field linearized approximation on the metric tensor $g_{\mu \nu}$ and we will derive the Einstein tensor $G_{\mu \nu}$ going through the following steps:
\begin{itemize}

\item[(a)] Calculate the Christoffel symbol
\begin{equation}
\label{christoffel_symbol}
:\Gamma ^\alpha _{\beta \gamma}: = 
\dfrac{1}{2} g^{\alpha ^\rho} \qty(
\partial_\beta g_{\gamma \rho} + 
\partial_\gamma g_{\rho \beta} -
\partial_\rho g_{\beta \gamma}
)
\end{equation}
where $\partial_\mu$ means the partial derivative $\partial/\partial x^{\mu}$.
\item[(b)] Calculate the Riemann curvature tensor
\begin{equation}
\label{riemann_tensor}
: R^\alpha _{\beta \gamma \sigma}:
=
:\Gamma ^\alpha _{\gamma \lambda}:\,
:\Gamma ^\lambda _{\sigma \beta}:
-
:\Gamma ^\alpha _{\sigma \lambda}:\,
:\Gamma ^\lambda _{\gamma \beta}:
+
\partial_\gamma :\Gamma^\alpha _{\sigma \beta}: 
-
\partial_\sigma :\Gamma^\alpha _{\gamma \beta} :
\end{equation}

\item[(c)] Obtain the Ricci tensor and the Ricci scalar from the Riemann curvature tenor
\begin{equation}
\label{ricci}
R_{\mu \nu} = :R^\alpha _\mu \alpha _\nu: \hspace{1.5cm}
R=\eta^{\mu \nu} R_{\mu \nu} = :R^{\mu} _\mu:
\end{equation}

\end{itemize}

$G_{\mu \nu}$ and $T_{\mu \nu}$ are symmetric tensors, since $g_{\mu \nu}$ is symmetric.
So the Einstein's field equation is a set of non-linear second-order partial differential equations with $10$ linearly independent variables. \\
We show that the equation(\ref{einstein_eq}) leads to gravitational wave solutions if we use the \textbf{weak field linearized approximation}, wheich means that we treat the spacetime as nearly flat. 
Therefore, we assume the metric tensor $g_{\mu \nu}$ to be equal to the Minkwoski metric $\eta=\text{diag}(-1,+1,+1,+1)$ plus a small metric perturbation $h_{\mu \nu}$ :
\begin{equation}
\label{metric}
g_{\mu \nu} = \eta_{\mu \nu} + h_{\mu \nu}
\end{equation}
where the metric perturbation is symmetric and $\abs{h_{\mu \nu}}\ll 1$ for all $\mu$ and $\nu$. \\
The metric $g_{\mu \nu}$ is also used to lower and raise indeces, however, in linearized theory we consider only the first order approximation in $h_{\mu \nu}$. So, it is possible to raise and lower indeces using the Minkoswkian metric $\eta_{\mu \nu}$.\\ % we can add the estimation of the swarzschild metric the correction at the sun surface h - 10^-6
Taking into account the mentioned approximations we follow the described procedure to calculate the so-called linearized Einstein's field equation.
The Christoffel symbol is obtained keeping up to the first order in the perturbation $h_{\mu \nu}$
\[
:\Gamma ^\alpha _{\beta \gamma}: = 
\dfrac{1}{2} \eta^{\alpha \rho}
\qty(
\partial_\beta h_{\gamma \rho } 
+
 \partial_\gamma h_{\rho \beta}
-
\partial_\rho h _{\beta \gamma}
) 
\]
Thus, the Riemann curvature tensor becomes 
\setlength{\jot}{10pt}
\begin{eqnarray}
:R^\mu _{\beta \gamma \de }: &=&  \notag
\partial_\gamma :\Gamma ^\mu _{\de \beta}: -
\partial_\de :\Gamma ^\mu _{\gamma \beta}: 
\\ \notag
&=& 
\dfrac{1}{2} 
\qty[
\eta^{\mu \rho} 
\qty(
\partial_\gamma \partial_\de h_{\beta \rho} + %cancel
\partial_\gamma \partial_\beta h_{\de \rho} -
\partial_\gamma \partial_\rho h_{\beta \de}
)
-
\eta^{\mu \sigma} 
\qty(
\partial_\de \partial_\beta h_{\gamma \sigma} +
\partial_\de \partial_\gamma h_{\beta \sigma} - %cancel
\partial_\de \partial_\sigma h_{\beta \gamma}
)
]
\\ 
&=&
\label{reimann_curvature_tensor_first_order}
\dfrac{1}{2} \qty(
\partial_\gamma \partial_\beta :h ^\mu _{\de}: -
\partial_\gamma \partial^\mu h_{\beta \de} -
\partial_\de \partial_\beta :h^\mu _{ \gamma} : +
\partial_\de \partial^\mu h_{\beta \gamma}
)
\end{eqnarray}
where we neglected the first two terms in eq(\ref{riemann_tensor}) because they are second order terms. 
Contracting the first and the third indeces we get the Ricci tensor
\bea
R_{\beta \de} &=& 
\dfrac{1}{2} 
\qty(
\partial_\mu \partial_\beta :h ^\mu _{\de}: -
\partial_\mu \partial^\mu h_{\beta \de} -
\partial_\de \partial_\beta :h^\mu _{\mu } :+ 
\partial_\de \partial^\mu h_{\beta \mu}
) \\
&=&
\dfrac{1}{2} 
\qty(
\partial_\mu \partial_\beta :h^\mu _{\de} : -
\square h_{\beta \de} -
\partial_\de \partial_\beta h +
\partial_\de \partial^\mu h_{\beta \mu}
) 
\eea
where the trace of the perturbation is defined as $h=\eta^{\mu \nu} h_{\mu \nu}= :h^{\mu} _{\mu} :$, and the d'Alambertian opereator in flat space is $\square = \partial_\mu \partial^\mu$. \\
Contracting again to obtain the Ricci scalar yields
\begin{equation*}
\begin{aligned}
R &= \dfrac{1}{2} 
\qty(
\partial_\mu \partial^\nu :h ^\mu _{\nu}: -
\square :h^\beta _{\beta} : -
 \partial_\beta \partial^\beta h +
\partial_\nu \partial^\mu :h^\nu _\mu:
)
\\
&= 
\partial_\mu \partial_\nu h^{\mu \nu} - \square h
\end{aligned}
\end{equation*}
Therefore the Einstein tensor is
\begin{equation}
\label{einstein_tensor_approx}
G_{\beta \nu} = \dfrac{1}{2} 
\qty(
\partial_\mu \partial_\beta :h ^\mu _\nu: -
\square h_{\beta \nu} -
\partial_\nu \partial_\beta h +
\partial_\nu \partial^\mu h_{\beta \mu}
- 
\eta_{\beta \nu} \partial_\mu \partial_\lambda h^{\mu \lambda} - 
\eta_{\beta \nu} \square h
)
\end{equation}
If we define the \textbf{trace-reversed} perturbation 
\[
\bar{h} _{\mu \nu} = h_{\mu \nu} - \dfrac{1}{2} h \eta_{\mu \nu} \hspace{1.5cm} \bar{h}=\bar{h}^{\mu \nu} \eta_{\mu \nu} = - h
\]
we can simplify the equation(\ref{einstein_tensor_approx}).
Thus,
\begin{equation*}
\begin{aligned}
R_{\beta \nu} &= 
\dfrac{1}{2} 
\qty(
\partial_\mu \partial_\beta \bar{h}^{\mu}\! _{\nu } -
\square \bar{h}_{\beta \nu} -
\cancel{\partial_\nu \partial_\beta h} +
\partial_\nu \partial^\mu \bar{h}_{\beta \mu}
+
\cancel{\dfrac{1}{2} \eta_{\nu \mu} \partial^\mu \partial_\beta h}
-
\dfrac{1}{2} \eta_{\beta \nu} \square h 
+
\cancel{\dfrac{1}{2} \eta_{\beta \mu} \partial_\nu \partial^\mu h}
) \\
&=
\dfrac{1}{2} 
\qty(
\partial_\mu \partial_\beta \bar{h}^{\mu} \!_{\nu } -
\square \bar{h}_{\beta \nu}
+
\partial_\nu \partial^\mu \bar{h}_{\beta \mu}
-
\dfrac{1}{2} \eta_{\beta \nu} \square h 
)
\end{aligned}
\end{equation*}
And contracting the above tensor, we obtain
\begin{equation*}
\begin{aligned}
R &=
\partial_\mu \partial_\beta \bar{h}^{\mu}\! _{\nu } +
 \dfrac{1}{2} \eta^{\mu \nu} \partial_\mu \partial_\nu h - \square h
\end{aligned} = 
\partial_\mu \partial_\nu \bar{h}^{\mu \nu} - 
\dfrac{1}{2} \square h
\end{equation*}
So the Einstein's tensor expressed as a function of $\bar{h} _{\mu \nu}$ is
\begin{equation}
\label{einstein_tensor_tracereversed}
G_{\beta \nu} = \dfrac{1}{2} 
\qty(
\partial_\mu \partial_\beta \bar{h}^\mu \! _{ \nu} -
\square \bar{h}_{\beta \nu}
+
\partial_\nu \partial^\mu \bar{h}_{\beta \mu}
- \eta_{\mu \nu} \partial_\mu \partial_\nu \bar{h}^{\mu \nu}
)
\end{equation}
This expression can be simplified further by choosing an appropriate gauge transformation.
Using the \textbf{Lorenz gauge} condition 
\begin{equation}
\label{lorentz_gauge}
\partial_{\mu} \bar{h}\, ^{\mu \nu} = 0 
\end{equation}
 the Einstein's tensor of equation(\ref{einstein_tensor_tracereversed}) becomes
\bea
G_{\beta \nu} = 
\dfrac{1}{2}
\qty(
\partial_\beta \partial_\mu \bar{h}^{\mu \alpha} \eta_{\alpha \nu}
+ 
\partial_\nu \partial_\mu \bar{h}^{\mu \alpha} \eta_{\alpha \beta}
-
\eta_{\mu \nu} \partial_{\nu} \partial_{\mu} \bar{h}^{\mu \nu}
-
\square \bar{h}_{\beta \nu}
) =
-\dfrac{1}{2} \square \bar{h}_{\beta \nu}
\eea
The linearized Einstein's field equation is
\begin{equation}
\label{linear_einstein_eq}
\square \bar{h}_{\beta \nu} = - 16 \pi\,  T_{\beta \nu} 
\end{equation}
The energy-momentumn tensor $T_{\beta \nu}$ is null in vacuum so the linearized Einstein's equation in vacuum assumes the form of the wave equation in a tensorial form
\begin{equation}
\label{wave_equation}
\square \bar{h}_{\beta \nu} =0  
\end{equation}
The above equation shows that the trace-reversed metric perturbation propagate as a wave distorting a flat spacetime.\\
The simplest solution to the linearized Einstein's equation(\ref{wave_equation}) is a plane wave
\[
\bar{h}_{\beta \nu} = A_{\beta \nu} \, \exp\qty(i\, k_{\alpha} x^{\alpha})
\]
where $A_{\beta \nu}$ is called \textbf{amplitude tensor} and it is symmetric, since $\bar{h}_{\mu \nu}$ is symmetric.\\
Substitution of the plane wave solution into equation(\ref{wave_equation}) implies that $k_{\alpha} k^{\alpha} =0$, so $k^{\alpha}$ is a null four vector. 
Therefore, the plane wave solution is a gravitational wave which travels at the speed of light in the spatial direction $\vb{k} = (k^{1},k^{2},k^{3})/k^{0}$ and with frequency $\omega=k^{0} $, i.e. $\bar{h}_{\beta \nu} = A_{\beta \nu} \, \exp\qty[i\, (\vb{k} \cdot \vb{x} - \omega t)]$.
Furthermore, any $\bar{h}_{\mu \nu}$ satisfying the linearized Einstein's field equation(\ref{wave_equation}) in vacuum describes a \textbf{gravitational wave} propagating at the speed of light, and it can be Fourier-expanded as a superposition of plane waves.\\

\subsection{Gauge transformations and GW polarizations}
A \textbf{gauge transformation} in linearized theory is defined as a transformation of the perturbation $h_{\mu \nu}$ into a new metric perturbation $h'_{\mu \nu}$, that satisfies
\begin{equation}
\label{gauge_transf}
h' _{\mu \nu} = h_{\mu \nu} + \partial_\mu \xi _{\nu} + \partial_\nu \xi _{\mu}
\end{equation}
for a given vector field $\xi^\mu$.
Gauge transformations are particularly important because they leave the Riemann curvature tensor unchanged (up to the first order in $h_{\mu \nu}$), such that, the physical spacetime is unchanged (a simple proof can be found in \cite{carroll_spacetime_2003}). 
The invariance of the curvature under such transformations is analogous to the traditional gauge invariance of electromagnetism.\\
Assuming that the Einstein's field equation(\ref{linear_einstein_eq}) are valid everywhere the metric perturbation $h_{\mu \nu}$ contains: gauge degrees of freedom; physical, radiative degrees of freedom; and physical, non-radiative degrees of freedom tied to the matter source of the GW. \\
 It is possible to show that the linearized Einstein's equation can be written as 5 Poisson-type quations, plus a wave equation for the transverse-traceless components of the metric perturbation, which represents the radiative degrees of freedom \cite{flanagan_basics_2005,carroll_spacetime_2003}.\\
 Neverthless this procedure will manifestly demonstrate that the radiative degrees of freedom in spacetime are two, it is a cumbersome and long derivation.
Instead, we ignore the degrees of freedom tied to the matter setting $T_{\mu \nu} =0$ and we analyze only plane wave solutions of equation(\ref{wave_equation}):
\[
\bar{h}_{\mu \nu} = A_{\mu \nu} \, \exp\qty(i\, k_{\alpha} x^{\alpha})
\]
By using the Lorenz gauge and the transverse traceless gauge, we reduce progressively the number degrees of freedom of a plane wave from 10 to 2.\\

We want now to find the conditions on the parameter $\xi_{\mu}$ in order to satisfy the Lorenz gauge condition, that we used in the previous section. 
The initial metric perturbation $h_{\mu \nu}$ tranforms into $h'_{\mu \nu}$ if a gauge transformation is used. However, the new trace reversed metric $\bar{h}'_{\mu \nu}$ transforms as
\begin{eqnarray}
\bar{h}'_{\mu \nu} &=& 
h _{\mu \nu} + 
\partial_\mu \xi _\nu +
\partial_\nu \xi_\mu 
-\dfrac{1}{2} \eta _{\mu \nu}\,
\qty(
 h \, + \partial_\alpha \xi^\alpha +
\partial_\alpha \xi^\alpha  \notag
 )
 \\
 \label{h_bar_gauge}
 \bar{h}'_{\mu \nu} &=&
 \bar{h} _{\mu \nu} + 
\partial_\mu \xi _\nu +
\partial_\nu \xi_\mu 
- \eta _{\mu \nu}\,\partial_\alpha \xi^\alpha 
\end{eqnarray}
Imposing the Lorenz gauge $\partial_{\mu} \bar{h}'\, ^{\mu \nu}=\partial_{\mu} \bar{h}\, ^{\mu \nu} = 0 $ we obtain
\bea
\partial_{\mu} \bar{h}'\, ^{\mu \nu} &=&
\partial_{\mu} \bar{h}\, ^{\mu \nu}+
\partial_{\mu} \partial^\mu \xi ^\nu +
\partial_{\mu} \partial^\nu \xi^\mu -
\partial_{\mu} \eta ^{\mu \nu}\,\partial_\alpha \xi^\alpha 
\\
&=&
0 +
\square \xi^{\nu}+
\partial^\nu \partial_{\mu} \xi^\mu -
\partial^{\nu} \partial_\alpha \xi^\alpha =0
\eea
Any metric perturbation $h_{\mu \nu}$ can therefore be put into a Lorenz gauge by using transformations that satisfy
\[
\square \xi_\mu = 0
\]
The plane wave $\xi_{\mu} = C_{\mu} \exp[i k_{\alpha} x^{\alpha}]$ is a solution of the above equation and it generates a gauge transformation through the four arbitrary constants $C_{\mu}$. \\
The \textbf{Transverse-Traceless (TT) gauge} is the most convinient gauge for the analysis of the gravitational waves, and it is defined for a plane wave by the following conditions:
\begin{itemize}
\item[a)] The Lorenz gauge condition fixes four components of $A_{\mu \nu}$
\[
\partial^{\mu} \bar{h}_{\mu \nu}=A_{\mu \nu} k^{\nu} =0
\]
The amplitude tensor $A_{\mu \nu}$ and the four vector $k^{\mu}$ are orthogonal.

\item[b)] Three components of the aplitude tensor can be eliminated selecting $\xi_{\mu} = C_{\mu} \exp[i k_{\alpha} x^{\alpha}]$ so that $A^{\mu \nu} u_{\mu} =0$ for some chosen four velocity $u_{\mu}$. Three and not four components are fixed, since one firther constraint $k^{\mu} A_{\mu \nu} u^{\nu}$ needs to be satisfied.
 
\item[c)] One component of the aplitude tensor can be eliminated selecting $\xi_{\mu} = C_{\mu} \exp[i k_{\alpha} x^{\alpha}]$ so that $A^{\mu} _{\mu} = 0$.

\end{itemize}
This means that we have suffecient freedom to fix the values of 8 components of $A_{\mu \nu}$ from a), b) and c), hence, reducing the number of independent component from 10 to 2 \cite{bishop_extraction_2016}.
Note that $\bar{h} ^{\T} _{\mu \nu}= h^{\T} _{\mu \nu}$ from c).
\\
What does the TT gauge tell us about gravitational radiation ?\\
Let use consider a test a particle at rest with four-velocity $u^{\alpha} = (1,0,0,0)$ in a nearly flat spacetime. If we orient our spatial coordinate axes so that the a plane gravitational wave is travelling in the positive z-direction (equivalentely $x^{3}$ direction) $k^{\sigma} = (\omega, 0 ,0 ,\omega)$ the transverse traceless conditions becomes
\[ \left.
\begin{aligned}
A^{\T}_{\mu 0} \, \omega + A ^{\T}_{\mu 3} \, \omega &= 0 \\
A^{\T}_{0 \nu}  &= 0 \\
A^{\T} _{00} + A^{\T} _{11} +A^{\T} _{22} + A^{\T} _{33} &= 0
\end{aligned}
\;
\right\}
\quad \Rightarrow \quad
\begin{bmatrix}
0 & 0 & 0 & 0 \\
0 & A^{\T} _{11} & A_{12} ^{\T} & 0 \\
0 & A^{\T} _{12} & -A^{\T} _{11} & 0 \\
0 & 0 & 0 & 0 \\
\end{bmatrix}
\]
As a consequence of the transverse traceless gauge  the only non-zero component of the metric perturbation $\bar{h}^{\T}_{\mu \nu}$ are, respectively, the plus ($+$) and the cross ($\times$) polarization of the gravitational wave
\bea
\bar{h}^{\T} _{11} &=& -\bar{h}^{\T} _{22} \equiv h_{+} \\
\bar{h}^{\T} _{12} &=& \bar{h}^{\T} _{21} \equiv h_{\times}  
\eea
So, the plane wave solution in the TT gauge is:
\beq
\label{matrix_h_tt}
\bar{h}^{\T}_{\mu \nu} =
\begin{bmatrix}
0 & 0 & 0 & 0 \\
0 & h_{+} & h_{\times} & 0 \\
0 & h_{\times} & -h_{+} & 0 \\
0 & 0 & 0 & 0 \\
\end{bmatrix}
\eeq
where we express the real part of the solution with $x^{0}=t$ and $x^{3}=z$ as follow
\bea
h_{+} = A^{\T} _{11} \cos \qty(\omega(t-z)) \\
h_{\times} =  A^{\T} _{12} \cos \qty(\omega(t-z))
\eea
$h_{+}$ and $h_{\times}$ are the two independent polarizations of a gravitational wave and they completely characterize the gravitational wave solution. 
We finally found that the radiative degrees of freedom are only two and they are represented by $h_{+}$ and $h_{\times}$. \\

Generally, within any finite vacuum region it is always possible to find a gauge which is locally transverse and traceless \cite{flanagan_basics_2005}, that is, a guage which satisfies the following general conditions
\bea
h_{0 \nu} ^{\T} = 0 \\
\eta ^{\mu \nu} h_{\mu \nu} ^{\T} =0 \\
\partial_{\mu} h^{\mu \nu} _{\T} =0
\eea
The transverse traceless gauge does not only simplify the expression of the perturbation metric, but it also gives an important relation between the Riemann curvature tensor and the metric perturbation.
Since we have already calculated the Riemann curvature tensor in equation(\ref{reimann_curvature_tensor_first_order}) we recall the result taking into account the TT gauge conditions
\begin{eqnarray}
:R^\mu _{00 \sigma }: 
&=&
\dfrac{1}{2} \qty(
\partial_0 \partial_0 :h^{\T \mu} _{\sigma}: -
\partial_0 \partial^\mu h^{\T}_{0 \sigma} -
\partial_\sigma \partial_0 :h^{\T \,\mu} _{ 0}:  +
\partial_\sigma \partial^\mu h_{0 0} ^{\T}
) \notag
\\
&=&
\label{reimann_tensor_TT}
\dfrac{1}{2} \partial_0 \partial_0 :h^{\T \mu} _{\sigma}: 
\hspace{1.5cm} \text{using } h^{\T} _{\mu 0} =0 
\end{eqnarray}
The above result tells us that the curvature of spacetime is proportional to the 'acceleration' of the gravitational wave. 
Considering a plane wave we have
\[
 :R^\mu _{00 \sigma }: = -\dfrac{1}{2} \omega^2 :A^{\T \mu} _{\sigma}: \cos (\omega (t-z))
\]
where $\omega$ is the frequency of the plane wave. 
The curvature is proportional to the square of the frequency, in fact we expect a bigger curvature if the wave oscillates more times per second.
Neverthless we considered the simple model of a plane wave, we can say, naively, that the spacetime is more curved if the ripples of the GW are squeezed in a narrow time interval.
Analogously, the electromagnetic field of an oscillating electric dipole is proportional to the square of the frequency, in fact we expect a more intense  field if the charge oscillates more times per second.


